\documentclass[11pt]{article}

% Language setting
\usepackage[turkish]{babel}
\usepackage{pythonhighlight}

\usepackage[a4paper,top=2cm,bottom=2cm,left=2cm,right=2cm,marginparwidth=2cm]{geometry}

% Useful packages
\usepackage{amsmath}
\usepackage{graphicx}
\usepackage[colorlinks=true, allcolors=blue]{hyperref}
\usepackage{verbatim}
\usepackage{fancyhdr} % for header and footer
\usepackage{titlesec}
\usepackage{parskip}

\setlength{\parindent}{0pt}

\titleformat{\subsection}[runin]{\bfseries}{\thesubsection}{1em}{}

\pagestyle{fancy} % activate the custom header/footer

% define the header/footer contents
\lhead{\small{23BLM-4014 Yapay Sinir Ağları Ara Sınav Soru ve Cevap Kağıdı}}
\rhead{\small{Dr. Ulya Bayram}}
\lfoot{}
\rfoot{}

% remove header/footer on first page
\fancypagestyle{firstpage}{
  \lhead{}
  \rhead{}
  \lfoot{}
  \rfoot{\thepage}
}
 

\title{Çanakkale Onsekiz Mart Üniversitesi, Mühendislik Fakültesi, Bilgisayar Mühendisliği Akademik Dönem 2022-2023\\
Ders: BLM-4014 Yapay Sinir Ağları/Bahar Dönemi\\ 
ARA SINAV SORU VE CEVAP KAĞIDI\\
Dersi Veren Öğretim Elemanı: Dr. Öğretim Üyesi Ulya Bayram}
\author{%
\begin{minipage}{\textwidth}
\raggedright
Öğrenci Adı Soyadı: KAĞAN SAĞDIÇ\\ % Adınızı soyadınızı ve öğrenci numaranızı noktaların yerine yazın
Öğrenci No: 190401086
\end{minipage}%
}

\date{14 Nisan 2023}

\begin{document}
\maketitle

\vspace{-.5in}
\section*{Açıklamalar:}
\begin{itemize}
    \item Vizeyi çözüp, üzerinde aynı sorular, sizin cevaplar ve sonuçlar olan versiyonunu bu formatta PDF olarak, Teams üzerinden açtığım assignment kısmına yüklemeniz gerekiyor. Bu bahsi geçen PDF'i oluşturmak için LaTeX kullandıysanız, tex dosyasının da yer aldığı Github linkini de ödevin en başına (aşağı url olarak) eklerseniz bonus 5 Puan! (Tavsiye: Overleaf)
    \item Çözümlerde ya da çözümlerin kontrolünü yapmada internetten faydalanmak, ChatGPT gibi servisleri kullanmak serbest. Fakat, herkesin çözümü kendi emeğinden oluşmak zorunda. Çözümlerinizi, cevaplarınızı aşağıda belirttiğim tarih ve saate kadar kimseyle paylaşmayınız. 
    \item Kopyayı önlemek için Github repository'lerinizin hiçbirini \textbf{14 Nisan 2023, saat 15:00'a kadar halka açık (public) yapmayınız!} (Assignment son yükleme saati 13:00 ama internet bağlantısı sorunları olabilir diye en fazla ekstra 2 saat daha vaktiniz var. \textbf{Fakat 13:00 - 15:00 arası yüklemelerden -5 puan!}
    \item Ek puan almak için sağlayacağınız tüm Github repository'lerini \textbf{en geç 15 Nisan 2023 15:00'da halka açık (public) yapmış olun linklerden puan alabilmek için!}
    \item \textbf{14 Nisan 2023, saat 15:00'dan sonra gönderilen vizeler değerlendirilmeye alınmayacak, vize notu olarak 0 (sıfır) verilecektir!} Son anda internet bağlantısı gibi sebeplerden sıfır almayı önlemek için assignment kısmından ara ara çözümlerinizi yükleyebilirsiniz yedekleme için. Verilen son tarih/saatte (14 Nisan 2023, saat 15:00) sistemdeki en son yüklü PDF geçerli olacak.
    \item Çözümlerin ve kodların size ait ve özgün olup olmadığını kontrol eden bir algoritma kullanılacaktır. Kopya çektiği belirlenen vizeler otomatikman 0 (sıfır) alacaktır. Bu nedenle çözümlerinizi ve kodlarınızı yukarıda sağladığım gün ve saatlere kadar kimseyle paylaşmayınız.
    \item Bu vizeden alınabilecek en yüksek not 100'dür. Toplam aldığınız puan 100'ü geçerse, aldığınız not 100'e sabitlenecektir.
    \item LaTeX kullanarak PDF oluşturanlar öz geçmişlerine LaTeX bildiklerini de eklemeyi unutmasınlar :)
    \item Bu vizedeki soruların çözümleri ve tex dosyası için istediğiniz kadar sayıda Github repository'si oluşturabilirsiniz. Sadece yukarıda belirttiğim tarihlerde (14 Nisan 2023 saat 15:00 ile 15 Nisan 2023 saat 15:00 arasında) public yapmayı/halka açmayı ve Github profilinizi de öz geçmişinize eklemeyi unutmayın :)
    \item Bilgisayarınıza hiçbir program kurmadan, Overleaf, Google Colab, Kaggle gibi ortamlar üzerinden bu vizeyi tamamlamanız mümkün. İyi çalışmalar!!
\end{itemize}

(Ekstra 5 puan) \url{https://github.com/kagansgdc/Neural}

\newpage
\section{(Toplam 10 Puan) Hopfield Networks:}

\subsection{(2 Puan)} \textbf{ChatGPT’ye size Hopfield Network’ler ile ilgili detaylı bilgi vermesini isteyin. Sorduğunuz soruyu, ChatGPT'nin çıkartığı cevabı aşağı koyun. Bu cevapta anlamadığınız terimleri de ChatGPT’ye sorup, sorduğunuz soruları ve ChatGPT’nin verdiği cevapları da sırayla aşağı koyun.}

Hopfield ağları, yapay sinir ağları alanında bir tür geri beslemeli ağdır. Bu ağlar, bellek işlevlerini yerine getirmek için kullanılır. Adını John Hopfield'den almıştır, 1982'de yayınlanan bir makalede bu ağı tanımlamıştır.

Hopfield ağı, aslında bir enerji fonksiyonu veya bir maliyet fonksiyonu olarak düşünülebilir. Bu fonksiyon, her bir nöronun durumunu ve tüm ağın durumunu dikkate alır. Ağın en iyi durumunu elde etmek için, enerji fonksiyonu minimize edilmelidir.

Hopfield ağları, nöronların birbirleriyle tamamen bağlantılı olduğu bir ağdır. Bu ağın amacı, önceden öğrenilmiş bilgileri hatırlamak ve hatırlamak için kullanılan bir model oluşturmaktır.

Ağın çalışması, bir veri girdisi alarak, bu girdiyi önceden öğrenilmiş bilgilerle karşılaştırarak ve en yakın benzerliği olan bilgiyi hatırlayarak gerçekleştirilir. Eğer veri girdisi önceden öğrenilmiş bilgiye tam olarak benzemiyor, Hopfield ağı hatırlanan bilgiye en yakın benzerliği olan bilgiyi verir.

Hopfield ağı, öğrenme ve hatırlama olmak üzere iki temel aşamadan oluşur. Öğrenme aşamasında, Hopfield ağına hatırlamak istediğimiz bilgiler verilir ve ağ, bu bilgileri depolar. Hatırlama aşamasında ise, ağa bir girdi verilir ve ağ, bu girdinin en yakın benzerliği olan bilgiyi hatırlar.

Hopfield ağları, diğer yapay sinir ağlarına göre daha basit bir yapıya sahiptir ve öğrenme süreci daha hızlıdır. Ancak, bu ağların karmaşıklığı arttıkça öğrenme süreci de zorlaşabilir.

Hopfield ağları, birçok farklı uygulama alanında kullanılabilir. Örneğin, resim tanıma, optimizasyon, örüntü tanıma, veri sıkıştırma ve veri düzeltme gibi alanlarda kullanılabilir. Ancak, büyük veri setleri üzerinde kullanıldığında performans sorunları ortaya çıkabilir.

Hopfield ağları, asenkron veya senkron olarak çalışabilir. Asenkron çalışmada, bir nöronun durumunun güncellenmesi, diğer nöronların durumlarını etkilemeden gerçekleştirilir. Senkron çalışmada ise, tüm nöronlar aynı anda güncellenir.

\subsection{(8 Puan)} \textbf{ChatGPT’nin açıklamalarından ne anladığınızı buraya kendi kelimelerinizle özetleyin. Örneğin ``Hopfield Network nedir? Ne işe yarar? Neden bilmemiz gerekir bunu? Gerçek hayatta kullanılıyor mu?'' gibi. Anlamadığınız kısımlar varsa ve ChatGPT’nin açıklamaları yeterli gelmezse internet dahil farklı kaynaklardan araştırıp, bilgilerin doğruluğunu sorgulamakta serbestsiniz. Konuyu doğru anlayıp anlamamanız puana etki edecektir. Kullandığınız kaynakları da belirtin!}

Girdi ve çıktı katmanı aynı olan tüm nöronların birbiri ile bağlı olduğu ağ modelidir. Tüm ağ için ve her bağlantı için bağlantı değerleri bir fonksiyonla kaydedilir. Önce depolama(öğrenme süreci) vardır, hatırlanmasını istediğimiz bilgi öğretilir, ardından geri besleme için gelen girdinin benzerliği kontrol edilir veya en yakın bilgi geri döndürülür. Elimizdeki öğrenilmiş bilgiye yeni veri seti gönderilidiğinde aynısını veya yakınlığını kontrol eder. Yeni girdi örüntüsündeki değişimden örüntü enerjisini yeniden hesaplayarak enerji fonksiyonunu minimize eder. Senkron ve Asenkron çeşidi vardır, senkronda tüm nöronlar güncellenirken asenkronda ilgili nöron güncellenir. Nesnenin otomatik olarak tanınması, işlenmesi, görüntü işleme, resim tanıma, optimizasyon ve veri düzeltme gibi alanlarda kullanılır. Günümüzde sıklıkla kullanılmakta.

Kaynak : https://medium.com/   - ChatGPT

\section{(Toplam 10 Puan) Boltzman Machine:}

\subsection{(2 Puan)} \textbf{ChatGPT’ye size Boltzman Machine ilgili detaylı bilgi vermesini isteyin. Sorduğunuz soruyu, ChatGPT'nin çıkartığı cevabı aşağı koyun. Bu cevapta anlamadığınız terimleri de ChatGPT’ye sorup, sorduğunuz soruları ve ChatGPT’nin verdiği cevapları da sırayla aşağı koyun.}

Boltzmann Machine (BM), yapay sinir ağı alanında en çok kullanılan modellerden biridir. İstatistik fizikteki Boltzmann dağılımından esinlenen BM, geri beslemeli bir sinir ağıdır ve düğümler ve bağlantılar ile temsil edilir.

BM, iki tip nöron içerir: gizli nöronlar ve görünür nöronlar. Gizli nöronlar, ağ tarafından hesaplanan özellikleri temsil ederken, görünür nöronlar, ağa verilen girdiyi temsil eder. BM, veri analizi, öğrenme ve tahmin yapma gibi birçok alanda kullanılır.

BM, birçok özelliği ile diğer yapay sinir ağı modellerinden ayrılır. Bunlar arasında, çift yönlü bağlantılar, stokastik davranış ve Gibbs örnekleme yöntemi sayılabilir. Ayrıca, BM, nöronların etkileşimlerini bir enerji fonksiyonu kullanarak modeller.

Gibbs örnekleme yöntemi, istatistiksel modellemelerde kullanılan bir Monte Carlo yöntemidir. Bu yöntem, çok boyutlu bir dağılımın örneklem çekimini gerçekleştirmek için kullanılır. Özellikle, Bayes teoremi kullanılarak elde edilen çok boyutlu bir olasılık dağılımı üzerinde çalışır.

BM'nin öğrenme süreci, ağın belirli bir veri setini gördüğü ve bu verileri incelediği süreçtir. Ağ, veri setindeki özellikleri belirler ve enerji fonksiyonunu optimize etmek için gereken ağırlıkları ayarlar. BM'nin tahmin süreci, ağa bir girdi verilerek gerçekleştirilir. Ağ, enerji fonksiyonunu minimize ederek girdinin en olası durumunu belirler.

BM, derin öğrenme için önemli bir rol oynamaktadır. Özellikle, sınıflandırma ve görüntü tanıma gibi alanlarda başarılı sonuçlar elde edilmiştir. BM, ayrıca veri sıkıştırma ve özellik çıkarımı gibi diğer birçok uygulama için de kullanılabilir.

BM'nin temel avantajları arasında, daha geniş bir veri seti kullanarak daha iyi sonuçlar elde edebilmesi, yeni veri örnekleriyle iyi bir şekilde genelleme yapabilmesi, ve insanların düşünme şeklini daha iyi yansıtabilmesi yer alır.

Ancak, BM'nin temel zorlukları arasında, hesaplama karmaşıklığı ve ağın büyüklüğü yer alır. BM, hesaplama gücüne ihtiyaç duyar ve büyük boyutlu veri setleri ile çalıştığında hesaplama yükü artar. Ayrıca, ağın öğrenmesi ve doğru sonuçlar üretmesi için yeterli sayıda nöron ve bağlantıya ihtiyaç vardır.

\subsection{(8 Puan)} \textbf{ChatGPT’nin açıklamalarından ne anladığınızı buraya kendi kelimelerinizle özetleyin. Örneğin ``Boltzman Machine nedir? Ne işe yarar? Neden bilmemiz gerekir bunu? Gerçek hayatta kullanılıyor mu?'' gibi. Anlamadığınız kısımlar varsa ve ChatGPT’nin açıklamaları yeterli gelmezse internet dahil farklı kaynaklardan araştırıp, bilgilerin doğruluğunu sorgulamakta serbestsiniz. Konuyu doğru anlayıp anlamamanız puana etki edecektir. Kullandığınız kaynakları da belirtin!}

Girdi katmanı ve gizli katman ayrı olmak üzere ; Girdi katmanında veriler, gizli katmanda(aynı gizli katmandaki nöronlar birbiriyle bağlı değil) bu verilere yön veren nöronlar bulunur(veriden her nörona ağırlık değerleri, her nöronun bias değeri ve fonksiyonu vardır) ve bir enerji fonksiyonu vardır. Öğrenme sürecinde belirli bir veri seti incelenir fonksiyonun optimize olması için ağırlıklar ayarlanır. Verinin hangi özelliğinin nekadar önemli olduğuna bağlı olarak bu ağırlıklar değişebilir. Back propagation mantığı vardır bu ağ modeli kendi kendini eğitir fonksiyonun en olası haline gelmesi beklenir. Derste de gördüğümüz balık en,boy,kg bilgilerine göre fonksiyonun en olası sonucu verip balığın hangi sınıfa ait olduğu bilgisi örneği verilebilir. Her yeni veri setinde eğitilebilir. Daha fazla gizli katman ile karmaşıklığı artabilir. Sınıflandırma ve görüntü tanıma gibi alanlarda sıkça kullanılır.(Müşteri profili kestirimi)

Kaynak : ChatGPT

\section{(Toplam 10 Puan) Markov Assumption ve Markov Chain:}

\subsection{(2 Puan)} \textbf{ChatGPT’ye size Markov assumption ve Markov Chain ilgili detaylı bilgi vermesini isteyin. Sorduğunuz soruyu, ChatGPT'nin çıkartığı cevabı aşağı koyun. Bu cevapta anlamadığınız terimleri de ChatGPT’ye sorup, sorduğunuz soruları ve ChatGPT’nin verdiği cevapları da sırayla aşağı koyun.}

Markov assumption, bir olayın sadece son olaya bağlı olduğunu ve önceki olayların gelecekteki olayların tahmininde bir rol oynamadığını varsayar. Bu varsayım, Markov Chain adı verilen matematiksel bir modelin temelini oluşturur.

Markov Chain, rastgele değişkenlerin bir dizi oluşturduğu bir matematiksel modeldir. Her değişken, bir durumu temsil eder ve her durumun oluşma olasılığı, yalnızca önceki duruma bağlıdır. Bu nedenle, bir Markov Chain'in gelecekteki durumu, yalnızca mevcut durumla ilgilidir ve önceki durumların bilgisine ihtiyaç duyulmaz.

Markov Chain, birçok farklı alanda kullanılır. Örneğin, finansal modellerle, öngörü modelleriyle, doğal dil işlemeyle, bilgisayar bilimleriyle, fizikle ve mühendislikle ilgili problemlerle uğraşan kişiler tarafından kullanılır. Markov Chain, olayların zaman içindeki evrimini modellerken kullanılabilir.

Markov Chain'in özellikleri arasında, durumların ve geçiş olasılıklarının belirli bir süreklilik sergilemesi yer alır. Ayrıca, durumların birbirinden bağımsız olması ve herhangi bir durumun, diğer durumların tarihi hakkında bilgi sahibi olmaması gerekmektedir. Bu özellikler, Markov Chain'in sınırlı bir bellek kapasitesine sahip olduğu anlamına gelir.

Markov Chain'in en önemli özelliklerinden biri, geçerli durumun gelecekteki durumu belirleme olasılığının yalnızca mevcut durumla ilgili olmasıdır. Bu nedenle, Markov Chain'in geçmişteki durumlara dayanmaması ve herhangi bir durumun gelecekteki durumu tahmin ederken yalnızca mevcut durumla ilgilenmesi, bu modelin öngörü gücünü artırır.

Markov Chain'in öğrenme süreci, önceki durumlarla sonraki durumlar arasındaki geçiş olasılıklarının hesaplanmasıyla gerçekleştirilir. Bu geçiş olasılıkları, ağın veri setindeki özellikleri belirlemek için kullanılır. Öğrenme süreci tamamlandıktan sonra, ağ tahmin yapmak için kullanılabilir.

Markov Chain, birçok farklı alanda kullanılan etkili bir matematiksel modeldir. Markov Chain'in özellikleri, birçok uygulamada kullanılmasını sağlar. Ancak, modelin sınırlı bellek kapasitesi, bazı uygulamalarda sınırlamalara neden olabilir ve diğer modellerin kullanılmasını gerektirebilir.

\subsection{(8 Puan)} \textbf{ChatGPT’nin açıklamalarından ne anladığınızı buraya kendi kelimelerinizle özetleyin. Örneğin ``Markov assumption ve Markov Chain nedir? Ne işe yarar? Neden bilmemiz gerekir bunu? Gerçek hayatta kullanılıyor mu?'' gibi. Anlamadığınız kısımlar varsa ve ChatGPT’nin açıklamaları yeterli gelmezse internet dahil farklı kaynaklardan araştırıp, bilgilerin doğruluğunu sorgulamakta serbestsiniz. Konuyu doğru anlayıp anlamamanız puana etki edecektir. Kullandığınız kaynakları da belirtin!}

(Bildiğimiz Olasılık.) Bir olayın geçmiş duruma bakılmaksınız o an bulunduğu duruma göre gelecekteki durumunun belirlenebildiği bir modeldir.Ben bir yol ayrımına geldim, nereden geldiğim ne tarafı seçeceğimi etkilemiyor şu an bulunduğum duruma göre tercih yapacağım. Geçişler olasılık dağılımlarına göre gerçekleşir bir sonraki durum için rastgelelik vardır.Hava durumu tahminleme, finansal piyasalar, doğal dil işleme gibi yerlerde kullanılır. İhtimaller günlük yaşantımızı büyük ölçüde etkiler. Tahminimce doğal dil işlemede dil modeli oluştururken, kelime tahmini yaparken kullanılıyor olabilir.

Kaynak : ChatGPT, Wikipedia

\section{(Toplam 20 Puan) Feed Forward:}
 
\begin{itemize}
    \item Forward propagation için, input olarak şu X matrisini verin (tensöre çevirmeyi unutmayın):\\
    $X = \begin{bmatrix}
        1 & 2 & 3\\
        4 & 5 & 6
        \end{bmatrix}$
    Satırlar veriler (sample'lar), kolonlar öznitelikler (feature'lar).
    \item Bir adet hidden layer olsun ve içinde tanh aktivasyon fonksiyonu olsun
    \item Hidden layer'da 50 nöron olsun
    \item Bir adet output layer olsun, tek nöronu olsun ve içinde sigmoid aktivasyon fonksiyonu olsun
\end{itemize}

Tanh fonksiyonu:\\
$f(x) = \frac{exp(x) - exp(-x)}{exp(x) + exp(-x)}$
\vspace{.2in}

Sigmoid fonksiyonu:\\
$f(x) = \frac{1}{1 + exp(-x)}$

\vspace{.2in}
 \textbf{Pytorch kütüphanesi ile, ama kütüphanenin hazır aktivasyon fonksiyonlarını kullanmadan, formülünü verdiğim iki aktivasyon fonksiyonunun kodunu ikinci haftada yaptığımız gibi kendiniz yazarak bu yapay sinir ağını oluşturun ve aşağıdaki üç soruya cevap verin.}
 
\subsection{(10 Puan)} \textbf{Yukarıdaki yapay sinir ağını çalıştırmadan önce pytorch için Seed değerini 1 olarak set edin, kodu aşağıdaki kod bloğuna ve altına da sonucu yapıştırın:}

% Latex'de kod koyabilirsiniz python formatında. Aşağıdaki örnekleri silip içine kendi kodunuzu koyun
\begin{python}
import torch
# set seed
torch.manual_seed(1)
def tanhfunc(x):
    return (torch.exp(x)-torch.exp(-x))/(torch.exp(x)+torch.exp(-x))
def sigmoid(x):
    return 1 / (1 + torch.exp(-x))
# convert X matrix to tensor
X = torch.tensor([[1., 2., 3.], [4., 5., 6.]])

# define neural network architecture
inputlayer = X.shape[1]
hiddenlayer = 50
outputlayer = 1

hw = torch.randn(inputlayer, hiddenlayer)

hb = torch.zeros(hiddenlayer)

# define weights for output layer
ow = torch.randn(hiddenlayer, outputlayer)

# define bias for output layer
ob = torch.zeros(outputlayer)

# define activation functions

# perform forward propagation
hiddenfunc = tanhfunc(torch.matmul(X, hw) + hb)
outputfunc = sigmoid(torch.matmul(hiddenfunc, ow) + ob)

print(outputfunc)
\end{python}

tensor([[0.9997],
        [0.9976]])

\subsection{(5 Puan)} \textbf{Yukarıdaki yapay sinir ağını çalıştırmadan önce Seed değerini öğrenci numaranız olarak değiştirip, kodu aşağıdaki kod bloğuna ve altına da sonucu yapıştırın:}

\begin{python}
import torch
# set seed
torch.manual_seed(190401086)
def tanhfunc(x):
    return (torch.exp(x)-torch.exp(-x))/(torch.exp(x)+torch.exp(-x))
def sigmoid(x):
    return 1 / (1 + torch.exp(-x))
# convert X matrix to tensor
X = torch.tensor([[1., 2., 3.], [4., 5., 6.]])

# define neural network architecture
inputlayer = X.shape[1]
hiddenlayer = 50
outputlayer = 1

hw = torch.randn(inputlayer, hiddenlayer)

hb = torch.zeros(hiddenlayer)

# define weights for output layer
ow = torch.randn(hiddenlayer, outputlayer)

# define bias for output layer
ob = torch.zeros(outputlayer)

# define activation functions

# perform forward propagation
hiddenfunc = tanhfunc(torch.matmul(X, hw) + hb)
outputfunc = sigmoid(torch.matmul(hiddenfunc, ow) + ob)

print(outputfunc)
\end{python}

tensor([[0.9490],
        [0.9751]])

\subsection{(5 Puan)} \textbf{Kodlarınızın ve sonuçlarınızın olduğu jupyter notebook'un Github repository'sindeki linkini aşağıdaki url kısmının içine yapıştırın. İlk sayfada belirttiğim gün ve saate kadar halka açık (public) olmasın:}
% size ait Github olmak zorunda, bu vize için ayrı bir github repository'si açıp notebook'u onun içine koyun. Kendine ait olmayıp da arkadaşının notebook'unun linkini paylaşanlar 0 alacak.

\url{https://github.com/kagansgdc/Neural}

\section{(Toplam 40 Puan) Multilayer Perceptron (MLP):} 
\textbf{Bu bölümdeki sorularda benim vize ile beraber paylaştığım Prensesi İyileştir (Cure The Princess) Veri Seti parçaları kullanılacak. Hikaye şöyle (soruyu çözmek için hikaye kısmını okumak zorunda değilsiniz):} 

``Bir zamanlar, çok uzaklarda bir ülkede, ağır bir hastalığa yakalanmış bir prenses yaşarmış. Ülkenin kralı ve kraliçesi onu iyileştirmek için ellerinden gelen her şeyi yapmışlar, ancak denedikleri hiçbir çare işe yaramamış.

Yerel bir grup köylü, herhangi bir hastalığı iyileştirmek için gücü olduğu söylenen bir dizi sihirli malzemeden bahsederek kral ve kraliçeye yaklaşmış. Ancak, köylüler kral ile kraliçeyi, bu malzemelerin etkilerinin patlayıcı olabileceği ve son zamanlarda yaşanan kuraklıklar nedeniyle bu malzemelerden sadece birkaçının herhangi bir zamanda bulunabileceği konusunda uyarmışlar. Ayrıca, sadece deneyimli bir simyacı bu özelliklere sahip patlayıcı ve az bulunan malzemelerin belirli bir kombinasyonunun prensesi iyileştireceğini belirleyebilecekmiş.

Kral ve kraliçe kızlarını kurtarmak için umutsuzlar, bu yüzden ülkedeki en iyi simyacıyı bulmak için yola çıkmışlar. Dağları tepeleri aşmışlar ve nihayet "Yapay Sinir Ağları Uzmanı" olarak bilinen yeni bir sihirli sanatın ustası olarak ün yapmış bir simyacı bulmuşlar.

Simyacı önce köylülerin iddialarını ve her bir malzemenin alınan miktarlarını, ayrıca iyileşmeye yol açıp açmadığını incelemiş. Simyacı biliyormuş ki bu prensesi iyileştirmek için tek bir şansı varmış ve bunu doğru yapmak zorundaymış. (Original source: \url{https://www.kaggle.com/datasets/unmoved/cure-the-princess})

(Buradan itibaren ChatGPT ve Dr. Ulya Bayram'a ait hikayenin devamı)

Simyacı, büyülü bileşenlerin farklı kombinasyonlarını analiz etmek ve denemek için günler harcamış. Sonunda birkaç denemenin ardından prensesi iyileştirecek çeşitli karışım kombinasyonları bulmuş ve bunları bir veri setinde toplamış. Daha sonra bu veri setini eğitim, validasyon ve test setleri olarak üç parçaya ayırmış ve bunun üzerinde bir yapay sinir ağı eğiterek kendi yöntemi ile prensesi iyileştirme ihtimalini hesaplamış ve ikna olunca kral ve kraliçeye haber vermiş. Heyecanlı ve umutlu olan kral ve kraliçe, simyacının prensese hazırladığı ilacı vermesine izin vermiş ve ilaç işe yaramış ve prenses hastalığından kurtulmuş.

Kral ve kraliçe, kızlarının hayatını kurtardığı için simyacıya krallıkta kalması ve çalışmalarına devam etmesi için büyük bir araştırma bütçesi ve çok sayıda GPU'su olan bir server vermiş. İyileşen prenses de kendisini iyileştiren yöntemleri öğrenmeye merak salıp, krallıktaki üniversitenin bilgisayar mühendisliği bölümüne girmiş ve mezun olur olmaz da simyacının yanında, onun araştırma grubunda çalışmaya başlamış. Uzun yıllar birlikte krallıktaki insanlara, hayvanlara ve doğaya faydalı olacak yazılımlar geliştirmişler, ve simyacı emekli olduğunda prenses hem araştırma grubunun hem de krallığın lideri olarak hayatına devam etmiş.

Prenses, kendisini iyileştiren veri setini de, gelecekte onların izinden gidecek bilgisayar mühendisi prensler ve prensesler başkalarına faydalı olabilecek yapay sinir ağları oluşturmayı öğrensinler diye halka açmış ve sınavlarda kullanılmasını salık vermiş.''

\textbf{İki hidden layer'lı bir Multilayer Perceptron (MLP) oluşturun beşinci ve altıncı haftalarda yaptığımız gibi. Hazır aktivasyon fonksiyonlarını kullanmak serbest. İlk hidden layer'da 100, ikinci hidden layer'da 50 nöron olsun. Hidden layer'larda ReLU, output layer'da sigmoid aktivasyonu olsun.}

\textbf{Output layer'da kaç nöron olacağını veri setinden bakıp bulacaksınız. Elbette bu veriye uygun Cross Entropy loss yöntemini uygulayacaksınız. Optimizasyon için Stochastic Gradient Descent yeterli. Epoch sayınızı ve learning rate'i validasyon seti üzerinde denemeler yaparak (loss'lara overfit var mı diye bakarak) kendiniz belirleyeceksiniz. Batch size'ı 16 seçebilirsiniz.}

\subsection{(10 Puan)} \textbf{Bu MLP'nin pytorch ile yazılmış class'ının kodunu aşağı kod bloğuna yapıştırın:}

\begin{python}
kod_buraya = None
if kod_buraya:
    devam_ise_buraya = 0

print(devam_ise_buraya)
\end{python}

\subsection{(10 Puan)} \textbf{SEED=öğrenci numaranız set ettikten sonra altıncı haftada yazdığımız gibi training batch'lerinden eğitim loss'ları, validation batch'lerinden validasyon loss değerlerini hesaplayan kodu aşağıdaki kod bloğuna yapıştırın ve çıkan figürü de alta ekleyin.}

\begin{python}
kod_buraya = None
if kod_buraya:
    devam_ise_buraya = 0

print(devam_ise_buraya)
\end{python}

% Figure aşağıda comment içindeki kısımdaki gibi eklenir.
\begin{comment}
\begin{figure}[ht!]
    \centering
    \includegraphics[width=0.75\textwidth]{mypicturehere.png}
    \caption{Buraya açıklama yazın}
    \label{fig:my_pic}
\end{figure}
\end{comment}

\subsection{(10 Puan)} \textbf{SEED=öğrenci numaranız set ettikten sonra altıncı haftada ödev olarak verdiğim gibi earlystopping'deki en iyi modeli kullanarak, Prensesi İyileştir test setinden accuracy, F1, precision ve recall değerlerini hesaplayan kodu yazın ve sonucu da aşağı yapıştırın. \%80'den fazla başarı bekliyorum test setinden. Daha düşükse başarı oranınız, nerede hata yaptığınızı bulmaya çalışın. \%90'dan fazla başarı almak mümkün (ben denedim).}

\begin{python}
kod_buraya = None
if kod_buraya:
    devam_ise_buraya = 0

print(devam_ise_buraya)
\end{python}

Sonuçlar buraya

\subsection{(5 Puan)} \textbf{Tüm kodların CPU'da çalışması ne kadar sürüyor hesaplayın. Sonra to device yöntemini kullanarak modeli ve verileri GPU'ya atıp kodu bir de böyle çalıştırın ve ne kadar sürdüğünü hesaplayın. Süreleri aşağıdaki tabloya koyun. GPU için Google Colab ya da Kaggle'ı kullanabilirsiniz, iki ortam da her hafta saatlerce GPU hakkı veriyor.}

\begin{table}[ht!]
    \centering
    \caption{Buraya bir açıklama yazın}
    \begin{tabular}{c|c}
        Ortam & Süre (saniye) \\\hline
        CPU & kaç? \\
        GPU & kaç?\\
    \end{tabular}
    \label{tab:my_table}
\end{table}

\subsection{(3 Puan)} \textbf{Modelin eğitim setine overfit etmesi için elinizden geldiği kadar kodu gereken şekilde değiştirin, validasyon loss'unun açıkça yükselmeye başladığı, training ve validation loss'ları içeren figürü aşağı koyun ve overfit için yaptığınız değişiklikleri aşağı yazın. Overfit, tam bir çanak gibi olmalı ve yükselmeli. Ona göre parametrelerle oynayın.}

Cevaplar buraya

% Figür aşağı
\begin{comment}
\begin{figure}[ht!]
    \centering
    \includegraphics[width=0.75\textwidth]{mypicturehere.png}
    \caption{Buraya açıklama yazın}
    \label{fig:my_pic}
\end{figure}
\end{comment}

\subsection{(2 Puan)} \textbf{Beşinci soruya ait tüm kodların ve cevapların olduğu jupyter notebook'un Github linkini aşağıdaki url'e koyun.}

\url{www.benimgithublinkim.com}

\section{(Toplam 10 Puan)} \textbf{Bir önceki sorudaki Prensesi İyileştir problemindeki yapay sinir ağınıza seçtiğiniz herhangi iki farklı regülarizasyon yöntemi ekleyin ve aşağıdaki soruları cevaplayın.} 

\subsection{(2 puan)} \textbf{Kodlarda regülarizasyon eklediğiniz kısımları aşağı koyun:} 

\begin{python}
kod_buraya = None
if kod_buraya:
    devam_ise_buraya = 0

print(devam_ise_buraya)
\end{python}

\subsection{(2 puan)} \textbf{Test setinden yeni accuracy, F1, precision ve recall değerlerini hesaplayıp aşağı koyun:}

Sonuçlar buraya.

\subsection{(5 puan)} \textbf{Regülarizasyon yöntemi seçimlerinizin sebeplerini ve sonuçlara etkisini yorumlayın:}

Yorumlar buraya.

\subsection{(1 puan)} \textbf{Sonucun github linkini  aşağıya koyun:}

\url{www.benimgithublinkim2.com}

\end{document}